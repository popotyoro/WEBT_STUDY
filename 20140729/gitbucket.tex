\documentclass[dvipdfmx]{beamer}
\AtBeginDvi{\special{pdf:tounicode 90ms-RKSJ-UCS2}}
\usetheme{Madrid}
\title{GitHub(Gitbucket)でできること}
\author{鈴木耕介}
\institute[]{}
\date{2014/7/29}
\usepackage{graphicx}
\usepackage{ulem}

\renewcommand{\kanjifamilydefault}{\gtdefault}
\usepackage[deluxe, expert]{otf}

\AtBeginSection[]{
    \frame{\tableofcontents[currentsection, hideallsubsections]} %目次スライド
}

\begin{document}
\frame{\titlepage}

\section{はじめに}
\begin{frame}{}
\begin{center}
{\Huge みんな
お久しぶり!!}
\end{center}
\end{frame}

\begin{frame}
\begin{center}
\begin{Huge}
ところでみんな・・
\end{Huge}
\end{center}
\end{frame}

\begin{frame}
\begin{figure}
\includegraphics[scale=0.5,bb=620 400 300 272]{lib/photo/oni.jpg}
\end{figure}

\begin{center}
\begin{Huge}
\textbf{\color{red}コード書いてる??}
\end{Huge}
\end{center}
\end{frame}

\begin{frame}
\begin{figure}
\includegraphics[scale=0.35,bb=920 600 300 272]{lib/photo/aooni.jpg}
\end{figure}

\begin{center}
\vspace{4cm}
\begin{Huge}
\textbf{\color{red}ぼくは書いてません}
\end{Huge}
\end{center}
\end{frame}

\begin{frame}
\begin{center}
\begin{Huge}
コードレビューしてる(してもらってる)??
\end{Huge}
\end{center}
\end{frame}

\begin{frame}
\begin{center}
\begin{Huge}
してないorしてもらってない\\
そこの貴方
\end{Huge}
\end{center}
\end{frame}

\begin{frame}
\begin{center}

\begin{Huge}
コード書ける力は絶対に上がりません!(偉そう)
\end{Huge}
\end{center}
\end{frame}

\begin{frame}
\Large{今日は自分が書いたコードの風通しを良くしてくれるモノをご紹介}
\end{frame}


\section{GitHubとは?}
\begin{frame}{GitHubって?}
\begin{figure}
\includegraphics[scale=0.5,bb=20 20 30 72]{lib/photo/github.png}
\end{figure}
\begin{enumerate}
\item \Large{GitHub}

\begin{itemize}
\item 同期、他人、プロジェクトメンバーとコードを共有するための場所(Gitリポジトリ)
を提供しているサービス
\vspace{1cm}
\item 付随する各種サービスがすんばらしぃぃぃ!(後で説明)
\end{itemize}

\end{enumerate}
\end{frame}

\begin{frame}{GitHubが提供する機能}

\begin{enumerate}
\item \Large{Gitリポジトリ}
\begin{itemize}
\item Gitの説明は前前前前回くらいにやったので略
\vspace{1cm}
\item 社内に公開を制限するサービスも提供している
\end{itemize}
\end{enumerate}
\end{frame}

\begin{frame}{GitHubが提供する機能}
\begin{enumerate}
\item \Large{Issue}
\begin{itemize}
\item タスク管理機能
\vspace{1cm}
\item バグ管理やチケット駆動開発のような使い方もできる
\vspace{1cm}
\item Issueとコミットを紐付けることができる
\end{itemize}
\end{enumerate}
\end{frame}

\begin{frame}{GitHubが提供する機能}
\begin{enumerate}
\item \Large{Wiki}
\begin{itemize}
\item その名の通りWiki機能
\vspace{1cm}
\item 開発ルールやマニュアルなど記載できる
\end{itemize}
\end{enumerate}
\end{frame}

\begin{frame}{GitHuGitHubが提供する機能}
\begin{enumerate}
\item \Large{Pull Request}
\begin{itemize}
\item AさんがpushしたものをBさんに取り込み要求できる機能
\vspace{1cm}
\item BさんはAさんがpushしたコードの差分を比較・議論ができる
\end{itemize}
\end{enumerate}
\end{frame}

\begin{frame}{GitHubが提供する機能}
\begin{center}
\begin{Huge}
どうですかみなさん? \\
この便利な機能を分かっていただけました?
\end{Huge}
\end{center}
\end{frame}

\begin{frame}{GitHubが提供する機能}
\includegraphics[scale=0.9,bb=20 168 30 172]{lib/photo/nonomura.jpg}
\begin{Huge}
えっ?\\
分からん?
\end{Huge}
\end{frame}

\begin{frame}{GitHubが提供する機能}
\begin{center}
\begin{Huge}
実際に動かしてみましょう
\end{Huge}
\end{center}
\end{frame}

\section{Gitbucketを使ってデモをする}

\begin{frame}{Gitbucketって?}
\begin{enumerate}
\item 今回は諸事情によりGitbucketを使います
\begin{itemize}
\item GitHubの\sout{パクリ}Clone
\vspace{1cm}
\item OSSでjavaだけで動く
\end{itemize}
\end{enumerate}

\end{frame}

\section{まとめ}
\begin{frame}{まとめ}
\begin{center}
\begin{Huge}
いかかでしたか?
\end{Huge}
\end{center}
\end{frame}

\begin{frame}{まとめ}
\begin{center}
\begin{Huge}
使ってみたくなってきたでしょう?
\end{Huge}
\end{center}
\end{frame}

\begin{frame}{まとめ}
\begin{center}
\begin{Huge}
javaで動くので明日から現場で実践可能ですよ!
\end{Huge}
\end{center}
\end{frame}

\begin{frame}{まとめ}
\begin{enumerate}
\pause
\item \Large{Gitでソース管理}
\pause
\vspace{1cm}
\item \Large{コードを見るor見てもらう}
\pause
\vspace{1cm}
\item \Large{簡単なタスク管理として使う}

\pause
\begin{figure}
\includegraphics[scale=1.0,bb=20 10 40 117]{lib/photo/githubNeko.png}
\end{figure}
\vspace{-3cm}
\Huge{\color{red}G O O D!}
\end{enumerate}



\end{frame}

\end{document}